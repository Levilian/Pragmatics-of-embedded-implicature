%=====================================================================
%============================= packages ==============================

\usepackage{amsmath}
\usepackage{amssymb}
\usepackage{stmaryrd}
\usepackage{fancyhdr}
\usepackage{natbib}
\usepackage[normalem]{ulem}
\usepackage{examples-slim}
\usepackage{color}
\usepackage{xcolor}
\usepackage{graphicx}
\usepackage{float}
\usepackage{booktabs}
\usepackage{colortbl}
\usepackage{qtree}
\usepackage{ifthen}
\usepackage{caption}
\usepackage{subcaption}
\usepackage{multirow}
\definecolor{black}{HTML}{000000}
\usepackage[colorlinks, linkcolor=black, urlcolor=black, citecolor=black]{hyperref}

% Good colors for colorblind readers (apparently):
\definecolor{cbgreen}{HTML}{1B9E77}
\definecolor{cborange}{HTML}{D95F02}
\definecolor{cbpurple}{HTML}{7570B3}

\bibpunct[; ]{(}{)}{;}{a}{}{,}  % natbib citation style

\defcitealias{ChierchiaFoxSpector08}{CFS}
\newcommand{\CFS}{\citetalias{ChierchiaFoxSpector08}}
\defcitealias{Geurts:Pouscoulous:2009}{G\&S}
\newcommand{\GP}{\citetalias{Geurts:Pouscoulous:2009}}

%=====================================================================
%========================= cross-references ==========================

% Flexible sec/fig/tbl/def cross-refs.
\newcommand{\Secref}[1]{Sec.~\ref{#1}}
\newcommand{\secref}[1]{sec.~\ref{#1}}
\newcommand{\dashsecref}[2]{secs.~\ref{#1}--\ref{#2}}

\newcommand{\Defref}[1]{Def.~\ref{#1}}
\newcommand{\defref}[1]{def.~\ref{#1}}
\newcommand{\Defrefc}[2]{\Defref{#1}, clause~\ref{#2}}
\newcommand{\defrefc}[2]{\defref{#1}, clause~\ref{#2}}

\newcommand{\Figref}[1]{Fig.~\ref{#1}}
\newcommand{\figref}[1]{fig.~\ref{#1}}
\newcommand{\dashfigref}[2]{figs.~\ref{#1}--\ref{#2}}
\newcommand{\Tabref}[1]{Tab.~\ref{#1}}
\newcommand{\tabref}[1]{tab.~\ref{#1}}

% Examples:
\newcommand{\eg}[1]{(\ref{#1})}
\newcommand{\subeg}[2]{(\ref{#1}\ref{#2})}
\newcommand{\dblsubeg}[3]{(\ref{#1}\ref{#2},~\ref{#3})}
\newcommand{\dashsubeg}[3]{(\ref{#1}\ref{#2}--\ref{#3})}

% In-text citations
\newcommand{\posscitet}[1]{\citeauthor{#1}'s~(\citeyear{#1})}
\newcommand{\sposscitet}[1]{\citeauthor{#1}'~(\citeyear{#1})}
\newcommand{\possciteauthor}[1]{\citeauthor{#1}'s}
\newcommand{\spossciteauthor}[1]{\citeauthor{#1}'}
\newcommand{\pgposscitet}[2]{\citeauthor{#1}'s~(\citeyear{#1}:~#2)}
\newcommand{\secposscitet}[2]{\citeauthor{#1}'s~(\citeyear{#1}:~$\S$#2)}
\newcommand{\pgcitealt}[2]{\citealt{#1}:~#2}
\newcommand{\seccitealt}[2]{\citealt{#1}:~$\S$#2}
\newcommand{\pgcitep}[2]{(\citealt{#1}:~#2)}
\newcommand{\seccitep}[2]{(\citealt{#1}:~$\S$#2)}
\newcommand{\pgcitet}[2]{\citeauthor{#1}~(\citeyear{#1}:~#2)}
\newcommand{\seccitet}[2]{\citeauthor{#1}~(\citeyear{#1}:~$\S$#2)}

%=====================================================================
%============================ text styles ============================

\newcommand{\word}[1]{\emph{#1}}
\newcommand{\tech}[1]{\textbf{#1}}
\newcommand{\highlight}[1]{\uline{#1}}

% Gray table cell:
\newcommand{\graycell}[1]{{\cellcolor[gray]{.8}#1}}

%=====================================================================
%=============================== models ==============================

\newcommand{\set}[1]{\ensuremath{\left\{ #1 \right\}}}
\newcommand{\True}{\texttt{T}}
\newcommand{\False}{\texttt{F}}
\newcommand{\entails}{\sqsubseteq}
\newcommand{\evee}{\mathbin{\overline{\vee}}}
\newcommand{\tuple}[1]{\langle #1 \rangle}

\newcommand{\Reals}{\mathbb{R}}
\newcommand{\given}{\mid}
\newcommand{\Indicator}{\mathbb{I}}

\newcommand{\sem}[1]{\ensuremath{\llbracket#1\rrbracket}}
\newcommand{\States}{W}
\newcommand{\state}{w}
\newcommand{\Lex}{\mathcal{L}}
\newcommand{\LexSet}{\mathbf{L}}
\newcommand{\Messages}{M}
\newcommand{\Refinable}{\textit{Refinable}}
\newcommand{\msg}{m}
\newcommand{\Costs}{C}
\newcommand{\StatePrior}{P}
\newcommand{\LexPrior}{P_{\LexSet}}

\newcommand{\listenerZero}{l_{0}}
\newcommand{\speakerOne}{s_{1}}
%\newcommand{\listenerOne}{l_{1}}
\newcommand{\UncertaintyListener}{L}

\newcommand{\nullmsg}{\mathbf{0}}

\newcommand{\ALT}{\emph{ALT}}
\newcommand{\OALT}{\mathop{O\negthinspace_{\ALT}}}

\newcommand{\Grammar}{\mathcal{G}}
\newcommand{\Refine}{\mathcal{R}}

\newcommand{\world}[1]{\texttt{#1}}
\newcommand{\Worlds}{W}
\newcommand{\Domain}{D}

\newcommand{\Likert}{\emph{Likert}}

%=====================================================================
%============================ annotations ============================

\let\oldmarginpar\marginpar
\renewcommand{\marginpar}[1]{\oldmarginpar[\color{red}\raggedright\scriptsize #1]{\color{red}\raggedright\scriptsize #1}}

\newcommand{\mynote}[1]{{\color{red}#1}}

%=====================================================================
%============================== grammar ==============================

\newcommand{\playera}{\texttt{a}}     
\newcommand{\playerb}{\texttt{b}}     
\newcommand{\playerc}{\texttt{c}}
\newcommand{\Vt}{V$_{\text{T}}$}
\newcommand{\Vi}{V$_{\text{I}}$}

% For writing grammar definitions:
\newcommand{\gsem}[1]{\sem{\text{#1}}}

% Attempt to simplify quantifier meaning code:
\newcommand{\genericquantifier}[3][]{%
  \ifthenelse{\equal{#1}{cardinality}}%
  {$\set{\tuple{w, X, Y} : |\set{x : \tuple{w,x} \in X} #2 \set{y : \tuple{w,y} \in Y}| #3}$}
  {$\set{\tuple{w, X, Y} :  \set{x : \tuple{w,x} \in X} #2 \set{y : \tuple{w,y} \in Y}  #3}$}}
  
% Attempt to simplify proper name meaning code:
\newcommand{\genericpn}[1]{\set{\tuple{w, Y} : #1 \in \set{x : \tuple{w,x} \in Y}}}

\newcommand{\scalarlex}[1]{
  \begin{array}[c]{r@{ \ = \ }l}
    \sem{\word{scored}} & \set{#1} \\
    \sem{\word{aced}}   & \set{\tuple{\world{A}, \playera}}
  \end{array}}

\newcommand{\genericscalar}[9]{
  \setlength{\arraycolsep}{4pt}
  \begin{array}[c]{r r r r}
    \toprule
    & \world{N} & \world{S} & \world{A} \\
    \midrule
    \word{A scored} & #1 & #2 & #3 \\
    \word{A aced}   & #4 & #5 & #6 \\
    \nullmsg        & #7 & #8 & #9 \\
    \bottomrule
  \end{array}}

\newcommand{\scalarspeaker}[9]{
  \setlength{\arraycolsep}{2pt}
  \begin{array}[c]{r r r r}
    \toprule
    & \word{A scored} & \word{A aced} & \nullmsg \\
    \midrule
    \world{N}  & #1 & #2 & #3 \\
    \world{S}  & #4 & #5 & #6 \\
    \world{A}  & #7 & #8 & #9 \\
    \bottomrule
  \end{array}}